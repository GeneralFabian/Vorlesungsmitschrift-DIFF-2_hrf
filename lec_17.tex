% !TEX root = ./Vorlesungsmitschrift DIFF 2.tex  
\lecture{Do 18.06. 10:15}{}
\begin{satz}\label{homogene_dgl_loesungsraum}
  Sei \( I\subset \reals \) Intervall und \( \mathbb{K}=\reals \) oder \( \complexs \). Sei \( A\maps I\to \sqmatrices{n}{K} \) stetig. Berechne \( \homogenreddglloesungsraum{A} \), die Menge aller Lösungen von \( x'=A(t)\matrixmult x \). Es gilt: \( \homogenreddglloesungsraum; \) ist \( n \)-dimensionaler Unterraum über \( \mathbb{K} \).

  Für \( u_1,\ddots,u_k\in \homogenreddglloesungsraum{A} \) sind äquivalent
  \begin{eigenschaftenenumerate}
    \item \label{homogene_dgl_loesungsraum:loesungen_unabhaengig}\( u_1,\dotsc,u_k \) sind linear unabhängig über \( \mathbb{K} \).
    \item \label{homogene_dgl_loesungsraum:loesungen_unabhaengig_fuer_einen_wert} \texists \( t_0\in I \) \sd \( u_1(t_0),\dotsc,u_k(t_0)\in \mathbb{K}^n \) linear unabhängig über \( \mathbb{K} \) sind.
    \item \label{homogene_dgl_loesungsraum:loesungen_unabhaengig_fuer_alle_werte}\( u_1(t),\dotsc,u_k(t)\in \mathbb{K}^n \) sind linear unabhängig über \( \mathbb{K} \) für alle \( t\in I \).
  \end{eigenschaftenenumerate}
\end{satz}
\begin{proof}
  \begin{itemize}
    \item \( \homogenreddglloesungsraum{A} \) ist Vektorraum \tsubset VR aller Abbildungen: \( I\to \mathbb{K}^n \).
    \begin{enumerate}
      \item \( 0\in \homogenreddglloesungsraum{A} \) \checkmark
      \item \( u,v\in \homogenreddglloesungsraum{A} \), \( \lambda\in \mathbb{K} \)
      \begin{align*}
        \implies (u+\lambda v)'&=u'+(\lambda v)'=A u+\lambda\cdot A v\\
        &=A(u+\lambda v)\\
        \implies u+\lambda v&\in \homogenreddglloesungsraum{A}.
      \end{align*}
    \end{enumerate}
    \item \ref{homogene_dgl_loesungsraum:loesungen_unabhaengig_fuer_alle_werte} \timplies \ref{homogene_dgl_loesungsraum:loesungen_unabhaengig_fuer_einen_wert} \timplies \ref{homogene_dgl_loesungsraum:loesungen_unabhaengig}.
    \minisec{\ref{homogene_dgl_loesungsraum:loesungen_unabhaengig} \timplies \ref{homogene_dgl_loesungsraum:loesungen_unabhaengig_fuer_alle_werte}} Seien \( u_1,\dotsc, u_k \) linear unabhängig. Angenommen, \texists  \( t_0 \) \sd \( u_1(t_0),\dotsc,u_k(t_0)\in \mathbb{K}^n\) linear abhängig sind, \dh \texists  \( \lambda_1,\dotsc,\lambda_k\in \mathbb{K} \) \sd 
    \begin{equation*}
      \lambda_1 u_1(t_0)+\dotsb+\lambda_k u_k (t_0)=0.
    \end{equation*}
    Aber
    \begin{equation*}
      u\definedas \lambda_1 u_1+\dotsb+\lambda_k u_k \in \homogenreddglloesungsraum{A}
    \end{equation*}
    und
    \begin{equation*}
      u(t_0)=0\implies u(t)=0\quad \forall t
    \end{equation*}
    \contra VOR\@.
    \item \( \dim-{L_H(A)}=n \):\\
    Sei \( (e_1,\dotsc,e_n) \) die kanonische Einheitsbasis von \( \mathbb{K}^n \). \texists \( u_1,\dotsc,u_n\in \homogenreddglloesungsraum{A} \) mit \( u_j(t_0)=e_j \). Wegen \ref{homogene_dgl_loesungsraum:loesungen_unabhaengig_fuer_einen_wert} \timplies \ref{homogene_dgl_loesungsraum:loesungen_unabhaengig} sind sie linear unabhängig \timplies \( \dim-{\homogenreddglloesungsraum{A}}\geq n \). Es kann aber höchstens \( n \) linear unabhängige Vektoren in \( \homogenreddglloesungsraum{A} \) geben: Denn seien \( v_1,\dotsc,v_k \) linear unabhängig, dann ist (wegen \ref{homogene_dgl_loesungsraum:loesungen_unabhaengig} \timplies \ref{homogene_dgl_loesungsraum:loesungen_unabhaengig_fuer_alle_werte}) \( v_1(t_0),\dotsc,, v_k(t_0) \) für ein \( t_0\in I \) linear unabhängig in \( \mathbb{K}^n \) \timplies \( k\leq n \).
  \end{itemize}
\end{proof}
\begin{definition*}\index{Fundamentalsystem}
  Eine Basis von \( \homogenreddglloesungsraum{A} \) nennt man auch \emph{(Lösungs-) Fundamentalsystem} (für \( x'=Ax \)).
\end{definition*}
\begin{bemerkung*}
  Aus \thref{homogene_dgl_loesungsraum} folgt: Eine Teilmenge \( \Set{u_1,\dotsc,u_n}\subset \homogenreddglloesungsraum{A} \) ist genau dann Fundamentalsystem für \( x'=Ax \), wenn
  \begin{equation*}
    \det-{\Phi}(t_0)\neq 0\quad \text{für ein }t_0\in I,
  \end{equation*}
  wobei
  \begin{equation*}
    \Phi=\begin{pNiceMatrix} u_{11} & \Cdots & u_{n1} \\ \Vdots &  & \Vdots \\ u_{1n} & \Cdots & u_{nn} \end{pNiceMatrix}
  \end{equation*}
  (\( u_{ij}= \) \( j \)-te Komponentenfunktion der \( i \)-ten Funktion).
\end{bemerkung*}
\begin{beispiel*}
  \( A\maps \reals\to \sqmatrices{2}{\reals} \). \( A(t)=\begin{pNiceMatrix} 0 & -\alpha \\ \alpha & 0 \end{pNiceMatrix}=\Delta \), \( \alpha\in \reals \), \( x'=A\matrixmult x \).
  \begin{equation*}
    u_1(t)=\begin{pNiceMatrix} \Cos{\alpha t} \\ \Sin{\alpha t} \end{pNiceMatrix}\qquad u_2(t)=\begin{pNiceMatrix} -\Sin{\alpha t} \\ -\Cos{\alpha t} \end{pNiceMatrix}
  \end{equation*}
  sind Lösungen. Sie bilden ein Fundamentalsystem, denn 
  \begin{equation*}
    \determinant{\begin{pNiceMatrix} u_1(t) & u_2(t) \end{pNiceMatrix}}=1\neq 0\quad \forall t.
  \end{equation*}
  Dies bedeutet: \emph{Jede} Lösung der DGL ist eine Linearkombination von \( u_1 \) und \( u_2 \).
  \begin{beispiel*}
    Anfangsbedingung \( x(0)=\begin{pNiceMatrix} 1 \\ 2 \end{pNiceMatrix} \).
    \begin{align*}
      u(t)&=a_1 u_1(t)+a_2 u_2(t)\\
      u(0)&=\begin{pNiceMatrix} a_1 \\ a_2 \end{pNiceMatrix}\needed{=}\begin{pNiceMatrix} 1 \\ 2 \end{pNiceMatrix}\\
      \implies u(t)&=\begin{pNiceMatrix} \Cos{\alpha t}-2\Sin+{\alpha t} \\ \Sin{\alpha t}+2 \Cos{\alpha t} \end{pNiceMatrix}.
    \end{align*}
  \end{beispiel*}
\end{beispiel*}
\begin{satz}\label{inhomogene_dgl_loesungsraum}
  Sei \( I \) Intervall \tsubset \( \reals \), \( \mathbb{K}=\complexs \) oder \( \reals \). 
  \( A\maps I\to \sqmatrices{n}{\mathbb{K}} \)und \( b\maps I\to \mathbb{K}^n \) stetig. 
  Bezeichne \( \inhomogenreddglloesungsraum{A}{b} \) die Lösungsmenge des inhomogenen
  Systems \( x'=A(t)\matrixmult x+b(t) \). 
  Ist dann \( v_0\maps I\to \reals^n \) eine Lösung des inhomogenen Systems, so gilt 
  \begin{equation*}
    \inhomogenreddglloesungsraum{A}{b}=v_0+\homogenreddglloesungsraum{A}.
  \end{equation*}
\end{satz}
\begin{bemerkung*}
  Man erhält also \emph{alle} Lösungen als eine Linearkombination von Lösungen der homogenen Gleichung und Addition irgendeiner Lösung der inhomogenen (\enquote{spezielle Lösung}).
\end{bemerkung*}
\begin{proof}
  \begin{proofdescription}
    \item[\( \inhomogenreddglloesungsraum{A}{b}\subset v_0+\homogenreddglloesungsraum{A} \)]
    Sei also \( v\in \inhomogenreddglloesungsraum{A}{b} \). Dann gilt 
    \begin{equation*}
      (v-v_0)'=v'-v_0'=(Av+b)-(Av_0+b).
    \end{equation*}
    Also ist
    \begin{equation*}
      v-v_0\in \homogenreddglloesungsraum{A}\implies v=v_0+(v-v_0)\in v_0+\homogenreddglloesungsraum{A}.
    \end{equation*}
    \item[\( \inhomogenreddglloesungsraum{A}{b}\supset v_0+\homogenreddglloesungsraum{A} \)] Sei also \( v=v_0+u \) mit \( u\in \homogenreddglloesungsraum{A} \)
    \begin{equation*}
      \implies v'=v_0'+u'=A v_0+b+Au=Av+b.
    \end{equation*}
  \end{proofdescription}
\end{proof}
\begin{lemma}[Variation der Konstanten]\label{variation_der_konstanten_die_zweite}
  Sei eine inhomogenene Gleichung wie in \thref{inhomogene_dgl_loesungsraum} gegeben. Es sei \( \Phi=(u_1,\dotsc,u_n) \) eine Fundamentalsystem für \( \homogenreddglloesungsraum{A} \).

  Dann ist \( t\mapsto v(t)=\Phi(t)\matrixmult w_b(t) \) mit
  \begin{equation*}
    w_b(t)=\Integrate{\inverse{\Phi(s)}\matrixmult b(s)}{s,t_0,t}+C
  \end{equation*}
  eine Lösung der inhomogenen Gleichung. Hier ist \( \matrixmult \) Matrixmultiplikation und \( \inverse{\Phi}(s) \) ist die inverse Matrix.
\end{lemma}
\begin{proof}
  Ansatz \( v=\Phi\matrixmult w \).
  \begin{align*}
    v'&=\Phi'\matrixmult w+ \Phi\cdot w'\\
    &=(u_1',\dotsc,u_n')\matrixmult w+\Phi\matrixmult w'\\
    &=(\braceannotate{Au_1,\dotsc, Au_n}{=A\matrixmult \Phi})\matrixmult w+\Phi q..
     w'\\
    &\needed{=}A\matrixmult \Phi\matrixmult w+b
  \end{align*}
  \timplies \Beh (\( w=w_b \)).
  
\end{proof}
\begin{beispiel*}
  \( x'=\begin{pNiceMatrix} 0 & -1 \\ 1 & 0 \end{pNiceMatrix}x+\begin{pNiceMatrix} 0 \\ t \end{pNiceMatrix} \).
  \begin{align*}
    &\Phi(t)=\begin{pNiceMatrix} \Cos{t} & -\Sin{t} \\ \Sin{t} & \Cos{t} \end{pNiceMatrix}\qquad \inverse{\Phi(t)}=\begin{pNiceMatrix} \Cos{t} & \Sin{t} \\ -\Sin{t} & \Cos{t} \end{pNiceMatrix}\\
    \implies &\inverse{\Phi(s)}\matrixmult b(s)=\begin{pNiceMatrix} s\Sin{s} \\ s\Cos{s} \end{pNiceMatrix}\\
    \implies &w_b(t)\begin{aligned}[t]
      &=\Integrate{\begin{pNiceMatrix} s\Sin{s} \\ s\Cos{s} \end{pNiceMatrix}}{s,0,t}+C\\
      &=\begin{pNiceMatrix} \Sin{t}-t\Cos{t} \\ \Cos{t}+t\Sin{t} \end{pNiceMatrix}+C
    \end{aligned}\\
    \implies &v(t)=\Phi(t)\matrixmult w_b(t)=\begin{pNiceMatrix} -t \\ 1 \end{pNiceMatrix}
  \end{align*}
  ist eine Lösung der inhomogenen DGL (wähle \( C=0 \)). Anfangsbedingung \zb \( v(0)=\begin{pNiceMatrix} -1 \\ 1 \end{pNiceMatrix} \). Äquivalente Vorgehensweise: \( v(t) \) wie oben \( +\lambda_1 u_1+\lambda_2 u_2 \) oder \( w_b(t)=\begin{pNiceMatrix} \Sin{t}\Cdots \\ \Cdots \end{pNiceMatrix} +\begin{pNiceMatrix} C_1 \\ C_2 \end{pNiceMatrix}\). Dann ist \( v(t)=\begin{pNiceMatrix} -t \\ 1 \end{pNiceMatrix}+\braceannotate{C_1u_1+C_2u_2}{\Phi\matrixmult \begin{pNiceMatrix} C_1 \\ C_2 \end{pNiceMatrix}} \).
  \begin{equation*}
    v(0)=\begin{pNiceMatrix} 0 \\ 1 \end{pNiceMatrix} +C_1 \begin{pNiceMatrix} 1 \\ 0 \end{pNiceMatrix}+C_2 \begin{pNiceMatrix} 0 \\ 1 \end{pNiceMatrix}\implies C_1=-1.
  \end{equation*}
\end{beispiel*}
\begin{notation*}
  Man satz \( \begin{pNiceMatrix} -t \\ 1 \end{pNiceMatrix} \) ist eine \emph{spezielle Lösung} des inhomogenen Systems und
  \begin{equation*}
    \begin{pNiceMatrix} -t \\ 1 \end{pNiceMatrix}+C_1 \begin{pNiceMatrix} \Cos{t} \\ \Sin{t} \end{pNiceMatrix}+C_2 \begin{pNiceMatrix} -\Sin{t} \\ \Cos{t} \end{pNiceMatrix}
  \end{equation*}
  ist die \emph{allgemeine Lösung}.
\end{notation*}
Mit Hilfe des Satzes von der Reduktion der Ordnung übertragen wir nun das eben bewiesene auf Differentialgleichungen höherer Ordnung. 

Sei \( \mathbb{K}=\reals \) oder \( \complexs \). Eine homogene lineare DGL \( k \)-ter Ordnung ist eine Gleichung der Form
\begin{equation*}
  x^{(k)}+A_{k-1}(t)\matrixmult x^{(k-1)}+\dotsb+A_1(t)\matrixmult x'+A_0(t)\matrixmult x=0\tag{\( * \)}\label{eq:lineare_homogene_dgl_kter_ordnung}
\end{equation*}
mit \( A_j\maps I\to \sqmatrices{n}{\mathbb{K}} \) stetig. Eine inhomogene lineare DGL ist von der Form
\begin{equation*}
  x^{(k)}+A_{k-1}(t)\matrixmult x^{(k-1)}+\dotsb+A_1(t)\matrixmult x'+A_0(t)\matrixmult x=b(t)\tag{\( ** \)}\label{eq:lineare_inhomogene_dgl_kter_ordnung}
\end{equation*}
mit \( A_j \) wie oben und \( b\maps I\to \mathbb{K}^n \) stetig.

Eine Lösung ist eine \( k \)-fach differenzierbare Funktion \( u\maps I\to \mathbb{K}^n \) mit
\begin{equation*}
  u^{(k)}+A_{k-1}(t)\matrixmult u^{(k-1)}+\dotsb+A_1(t)\matrixmult u'+A_0(t)\matrixmult u=0 \text{ \bzw } b(t)\quad \forall t\in I.
\end{equation*}
\begin{satz}
  \begin{enumerate}
    \item Die Menge aller Lösungen eines homogenen linearen Systems wie in \eqref{eq:lineare_homogene_dgl_kter_ordnung}, \( \inhomogendglloesungsraum{k}{A_0,\dotsc,A_{k-1}}{b} \), ist ein \( n\cdot k \)-dimensionaler Vektorraum.
    \item Für die Menge aller Lösungen eines homogenen linearen Systems wie in \eqref{eq:lineare_homogene_dgl_kter_ordnung},
    \begin{equation*}
      \inhomogendglloesungsraum{k}{A_0,\dotsc,A_{k-1}}{b},
    \end{equation*}
    gilt:
    \( \inhomogendglloesungsraum;{k}=v_0+\homogendglloesungsraum;{k} \) für eine beliebige Lösung \( v_0 \) des inhomogenen Systems.
    \item Ein \( k \)-Tupel 
    \( u_1,\dotsc,u_k\in \homogendglloesungsraum;{k} \)
     ist genau dann linear unabhängig, wenn für ein \( t_0\in I \) (somit für alle \( t\in I \)) die \emph{Wroński-Determinante} \index{Wroński-Determinante}
    \begin{equation*}
      \wronskideterminant{t_0}=\determinant{\begin{pNiceMatrix}
        u_1(t_0) & u_2(t_0) & \Cdots & u_k(t_0) \\
        u_1'(t_0) & \Vdots &  & u_k'(t_0) \\
        \Vdots &  & & \Vdots\\
        u_1^{(k-1)}(t_0) & \Cdots & & u_k^{(k-1)}(t_0)
      \end{pNiceMatrix}}
    \end{equation*}
    nicht \( 0 \) ist.
  \end{enumerate}
\end{satz}
\begin{proof}
  Das System 
  \begin{align*}
    x_0'&=x_1\\
    x_1'&=x_2\\
    &\vdots\\
    x_{k-2}'&=x_{k-1}\\
    x_{k-1}'&=-A_0(t)x_0-\dotsb-A_{k-1}(t)x_{n-1}+b(t)
  \end{align*}
  ist nach \thref{reduktion_der_ordnung} äquivalent zu \eqref{eq:lineare_inhomogene_dgl_kter_ordnung} \bzw für \( b(t)=0 \) \eqref{eq:lineare_homogene_dgl_kter_ordnung}. Jeder Lösung \( u\maps I\to \mathbb{K}^n \) von \eqref{eq:lineare_homogene_dgl_kter_ordnung} \bzw \eqref{eq:lineare_inhomogene_dgl_kter_ordnung} entspricht einer Lösung des obigen Systems (mit \( b=\explain{\text{Nullfunktion}}{\underline{0}} \) \bzw \( b\neq \underline{0} \))
  \begin{equation*}
    \begin{pNiceMatrix} u \\ u' \\ \Vdots \\ u^{(k-1)} \end{pNiceMatrix}  
  \end{equation*}
  und umgekehrt ist \( u_0 \) Lösung von \eqref{eq:lineare_homogene_dgl_kter_ordnung} \bzw \eqref{eq:lineare_inhomogene_dgl_kter_ordnung}, wenn
  \begin{equation*}
    \begin{pNiceMatrix} u \\ u' \\ \Vdots \\ u^{(k-1)} \end{pNiceMatrix}  
  \end{equation*}
  Lösung des obigen Systems ist (mit \( b=\underline{0} \) \bzw \( b\neq \underline{0} \)). Somit folgt die Behauptung aus \ref{homogene_dgl_loesungsraum} und \ref{inhomogene_dgl_loesungsraum}.
\end{proof}
\begin{bemerkung*}
  Auch \ref{variation_der_konstanten_die_zweite} (Variation der Konstanten) überträgt sich direkt auf \eqref{eq:lineare_inhomogene_dgl_kter_ordnung}.
\end{bemerkung*}
\begin{notation*}
  Eine Basis von \( \homogendglloesungsraum{k}{A_0,\dotsc,A_{k-1}} \) heißt wieder \emph{Fundamentalsystem.}
\end{notation*}
\begin{beispiel*}
  \( x''-\frac{1}{2t}x'+\frac{1}{2t^2}x=0 \) auf \( I=\reals_{>0} \). Wir raten 2 Lösungen \( u_1(t)=t \) und \( u_2(t)=\sqrt{t} \). Probe:
  \begin{gather*}
    0-\frac{1}{2t}1+\frac{1}{2t^2}t=0\logicspace \checkmark\\
    \frac{1}{2}\p*{ -\frac{1}{2t\sqrt{t}} }-\frac{1}{2t}\frac{1}{2\sqrt{t}}+\frac{1}{2t^2}\sqrt{t}=0\logicspace \checkmark.\\[1ex]
    \wronskideterminant{t}=\determinant{\begin{pNiceMatrix} u_1(t) & U_2(t) \\ u_1'(t) & u_2'(t) \end{pNiceMatrix}}=\determinant{\begin{pNiceMatrix} t & \sqrt{t} \\ 1 & \frac{1}{2\sqrt{t}} \end{pNiceMatrix}}=\frac{-\sqrt{t}}{2}\neq 0\quad \forall t>0.
  \end{gather*}
  \timplies Die allgemeine Lösung der DGL ist
  \begin{equation*}
    u(t)=e_1 t+ c_2 \sqrt{t}
  \end{equation*}
  mit \( c_j\in \reals \) (oder \( \complexs \)).
\end{beispiel*}
Was tut man, wenn man nur eine Lösung raten kann? Im Fall \( k=2 \) und \( n=1 \) führt folgendes Verfahren auf eine zweite Lösung:
\begin{lemma}\label{zweier_dgl_zweite_loesung}
  Sei \( I \) Intervall, \( a_0,a_1\maps I\to \mathbb{K} \) (\( \mathbb{K}=\reals \) oder \( \complexs \)) stetig. Sei \( u\maps I\to \mathbb{K} \) eine Lösung von
  \begin{equation*}
    x''+a_1(t)x'+a_0(t)x=0\tag{\( * \)}\label{eq:homogen_dgl_2ter}.
  \end{equation*}
  Auf \( J\subset I \) gelte \( u(t)\neq 0 \). Dann ist \( v\maps J\to \mathbb{K} \), \( v=u\cdot w \) Lösung von \eqref{eq:homogen_dgl_2ter}, wenn für \( w gilt \)
  \begin{equation*}
    w''(t)+\parens*{2\frac{u'(t)}{u(t)}+a_1(t)}w'(t)=0\tag{\( ** \)}\label{eq:zweier_dgl_zweite_loesung_multiplikator_bedingung}.
  \end{equation*}
  Ist \( w \) auf \( J \) nicht konstant, sind \( v \) und \( u \) linear unabhängig.
\end{lemma}
\begin{bemerkung*}
  \( w' \) löst also die DGL \ordinalnum{1} Ordnung
  \begin{equation*}
    y'+\parens*{2\frac{u'(t)}{u(t)}+a_1(t)}y=0,
  \end{equation*}
  also ist
  \begin{align*}
    w'(t)&=C_0 \exponential*{-\Integrate{2\frac{u'(s)}{u(s)}+a_1(s)}{s,t_0,t}}\\
    &=C\frac{1}{u(t)^2}\exponential*{-\Integrate{a_1(s)}{s,t_0,t}},
  \end{align*}
  woraus man ein \( w \) bestimmen kann, dass \eqref{eq:zweier_dgl_zweite_loesung_multiplikator_bedingung} löst.
\end{bemerkung*}
\begin{proof}[Beweis von \ref{zweier_dgl_zweite_loesung}]
  \begin{align*}
    v'&=u'w+uw'\\
    v''&=u''w+2u'w'+uw''\\
    &=-a_1u'w-a_0uw+2u'w'+uw''\\
    \implies v''+a_1'+a_0v&=2u'w'+a_1uw'+uw''=0,
  \end{align*}
  wenn \( w \) \eqref{eq:zweier_dgl_zweite_loesung_multiplikator_bedingung} erfüllt.

  Die \ordinalnum{2} Behauptung ist klar.
\end{proof}
\begin{beispiel*}
  Nochmal
  \begin{equation*}
    x''-\frac{1}{2t}x'+\frac{1}{2t^2}x=0
  \end{equation*}
auf \( I=\reals_{>0} \). \( u(t)=t \) und bestimmen \( v \) wie oben:
\begin{align*}
  w'(t)&=\frac{1}{u(t)^2}\exponential*{-\Integrate{a_1(s)}{s,t_0,t}}\\
  &=\tilde{C}\frac{1}{t^2}\sqrt{t}=\tilde{C}\frac{1}{t^{\frac{3}{2}}}\\
  w(t)&=\tilde{C}\frac{1}{\sqrt{t}}+C'\implies v=\sqrt{t}.
\end{align*}
(Man lässt hier oft die Konstanten einfach weg).
\end{beispiel*}
\begin{generalthm}[Besondere DGL'n mit \( k=2 \), \( n=1 \)]\label{besondere_dlgs}
  (Achtung: andere Notation als bisher!)

  Zu \( n\in \naturals_0 \) betrachten wir die 
  \begin{description}
    \item[Legendresche DGL] für \( -1<x<1 \)
    \begin{equation*}
      (1-x^2)y''-2xy'+n(n+1)y=0.
    \end{equation*}
    \item[Hermitesche DGL] für \( x\in \reals \)
    \begin{equation*}
      y''-2xy'+2ny=0.
    \end{equation*}
    \item[Laguerresche DGL] für \( x>0 \)
    \begin{equation*}
      xy''+(1-x)y'+ny=0.
    \end{equation*}
  \end{description}
\end{generalthm}
\begin{lemma}
  Zu \( n\in \naturals_0 \) definiert man
  \begin{align*}
    \legendrepolynomial{n}{x}&=\frac{1}{2^n \factorial{n}}\odv*[n]{(x^2-1)^n}{x}&&\text{Legendre-Polynom der Ordnung \( n \)}\\
    \hermitepolynomial{n}{x}&=(-1)^n e^{x^2}\odv*[n]{e^{-x^2}}{x}&&\text{Hermite-Polynom der Ordnung \( n \)}\\
    \laguerrepolynomial{n}{x}&=e^x\odv*[n]{x^n e^{-x}}{x}&&\text{Laguerre-Polynom der Ordnung \( n \)}
  \end{align*}
  Bei den angegebene Funktionen handelt es sich tatsächlich um Polynome der Ordnung \( n \). Sie lösen die zugehörigen DGL'n.
\end{lemma}
\begin{proof}
  \begin{proofdescription}
    \item[\ordinalnum{1} \Beh] \( \legendrepolynomial;{n} \): \( \odv*[n]{}{x} \)(Polynom der Ordnung \( 2n \)) \teq Polynom von Ordnung \( \leq n \).
    \begin{equation*}
      \odv*[n]{\parens*{(x-1)^n(x+1)^n}}{x}=\sum_{l=0}^{n}\braceannotate{\text{Ordnung}=n-l}{\odv*[l]{(x-1)^n}{x}}\braceannotate{\text{Ordnung}=l}{\odv*[n-l]{(x+1)^n{x}}{x}}
    \end{equation*}
    \( \hermitepolynomial;{n} \):
    \begin{equation*}
      \odv*[n]{e^{-x^2}}{x}=\odv*[n-1]{\parens*{-2xe^{-x^2}}}{x}=\odv*[n-2]{\parens*{(-2x)^2e^{-x^2}+\odv*{(-2x)}{x}e^{-x^2}}}{x}
    \end{equation*}
    \timplies wenn alle \( n \) Ableitungen auf \( e^{-x^2} \) wirken, erhält man \( (-2x)^n e^{-x^2} \). Die anderen Beiträge sind kleineren Grads multipliziert mit \( e^{-x^2} \).

    \( \laguerrepolynomial;{n} \): genauso.
    \item[\ordinalnum{2} \Beh] Nur für die Hermite-Polynome: Setze \( y(x)=e^{x^2}\odv*[n]{e^{-x^2}}{x} \). Es gilt \( e^{-x^2}y(x)=\odv*[n]{e^{-x^2}}{x} \).
    \begin{equation*}
      \implies \odv*[2]{\parens*{e^{-x^2}y(x)}}=\odv*[n+2]{e^{-x^2}}{x}=\odv*[n+1]{\parens*{-2xe^{-x^2}}}{x}.
    \end{equation*}
    Induktion:
    \begin{equation*}
      (xe^{-x^2})^{(n+1)}=x(e^{-x^2})^{(n+1)}+(n+1)(e^{-x^2})^{(n)}.
    \end{equation*}
    \begin{subproof}
      IA\@: \( \odv*{\parens*{xe^{-x^2}}}{x}=e^{-x^2}+x(e^{-x^2})' \).
      \( n\to n+1 \):
      \begin{align*}
        (xe^{-x^2})^{(n+1)}&=\parens*{\parens*{xe^{-x^2}}^{(n)}}'\\
        &\underset{\text{IV}}{=}\parens*{x\parens{e^{-x^2}}^{(n)}+n\parendiff+{e^{-x^2}}{n+1}}'\\
        &=\parens*{e^{-x^2}}^{(n)}+x\parens*{e^{-x^2}}^{(n+1)}+n\parens{e^{-x^2}}^{(n)}.
      \end{align*}
    \end{subproof}
    \begin{align*}
      \implies (e^{-x^2}y(x))^{(2)}&=-2x\parens*{e^{x^2}}^{(n+1)}-2(n+1)\parens*{e^{-x^2}}^{(n)}\\
      &=-2x\odv*{\parens*{e^{-x^2}y(x)}}{x}-2(n+1)e^{-x^2}y(x)\\
      &=e^{-x^2}(4x^2y(x)-2y'(x)-2(n+1)y(x)).
    \end{align*}
    Andererseits:
    \begin{gather*}
      \parendiff+*{e^{-x^2}y(x)}{2}=e^{-x^2}y''(x)+2(e^{-x^2})'y'(x)+\parendiff+*{e^{-x^2}}{2}y(x)\\
      =e^{-x^2}\parens*{y''(x)-4xy'(x)+(4x^2-2)y(x)}\\
      \implies y''(x)-2xy'(x)+2ny(x)=0-
    \end{gather*}
    \( \laguerrepolynomial;{n},\legendrepolynomial;{n} \) ähnlich.
  \end{proofdescription}  
\end{proof}
Ein Fundamentalsystem für diese Gleichungen (zu gegebenen \( n \)) findet man \zb wieder mit \thref{zweier_dgl_zweite_loesung}.

\begin{beispiel*}
  Für \( n=1 \) und die Legendre-Gleichung
  \begin{align*}
    w'(x)&=\exponential*{-\Integrate{\frac{2}{s}}{s}+\Integrate{\frac{2s}{1-s^2}}{s}}\\
    &=\frac{1}{x^2}\frac{1}{(1-x^2)}\\
    \implies w(x)&=-\frac{1}{x}+\frac{1}{2}\Ln{\frac{1+x}{1-x}}.
  \end{align*}
\end{beispiel*}
